\chapter{Appendice - Esempi noti}

\section{Paradosso del compleanno}

Il paradosso afferma che la probabilità che almeno due persone in un gruppo compiano gli anni lo stesso giorno è largamente superiore a quanto potrebbe dire l'intuito. Si calcola inizialmente la probabilità che, nel gruppo, non esistano due compleanni uguali.

\begin{center}
    $\prob(A) = \frac{364!}{365^{n-1} \cdot (365 - n)!}$
\end{center}

Di questa probabilità si deve fare il complementare.

\begin{center}
    $\prob(A) = 1 - \frac{364!}{365^{n-1} \cdot (365 - n)!}$
\end{center}
 
\section{Problema di Monty Hall}

 Il problema di Monty Hall è un famoso problema di teoria della probabilità, legato al gioco a premi statunitense Let's Make a Deal. Prende il nome da quello del conduttore dello show, Monte Halprin, noto con lo pseudonimo di Monty Hall. Il problema è anche noto come paradosso di Monty Hall, poiché la soluzione può apparire controintuitiva, ma non si tratta di una vera antinomia, in quanto non genera contraddizioni logiche.

 Nel gioco vengono mostrate al concorrente tre porte chiuse; dietro ad una si trova un'automobile, mentre ciascuna delle altre due nasconde una capra. Il giocatore può scegliere una delle tre porte, vincendo il premio corrispondente. Dopo che il giocatore ha selezionato una porta, ma non l'ha ancora aperta, il conduttore dello show – che conosce ciò che si trova dietro ogni porta – apre una delle altre due, rivelando una delle due capre, e offre al giocatore la possibilità di cambiare la propria scelta iniziale, passando all'unica porta restante; cambiare la porta migliora le chance del giocatore di vincere l'automobile, portandole da 1/3 a 2/3.

 Questo accade perchè inizialmente la probabilità che il concorrente abbia scelto una capra è maggiore rispetto alla probabilità cha abbia scelto l'automobile.