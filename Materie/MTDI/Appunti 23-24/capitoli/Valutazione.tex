\chapter{Cenni sulla valutazione dell'apprendimento}

Una rubrica \glitter{valutativa} non serve solamente per assegnare un voto a uno studente,
ma anche per \newfancyglitter{aiutarlo a capire} quali sono i suoi punti di \fancyglitter{forza} e di \fancyglitter{debolezza} per poter migliorare.

\nt{Lo studente deve essere parte attivà del processo di valutazione.}

\section{Conoscenze, abilità e competenze}

\dfn{Conoscenze}{
    Le \newfancyglitter{conoscenze} indicano i risultati delle informazioni assimilate.
}

\nt{Le conoscenze sono descritte come teoriche e/o pratiche.}

\dfn{Abilità}{
    Le \newfancyglitter{abilità} indicano la capacità di applicare le conoscenze.
}

\nt{Le abilità sono descritte come cognitive (uso del pensiero logico) e pratiche
(che richiedono l'uso di materiali).}

\dfn{Competenze}{
    Le \newfancyglitter{competenze} indicano la capacità di applicare le conoscenze e le abilità in un contesto specifico.
}

\nt{Le competenze sono descritte in termini di responsabilità e autonomia.}