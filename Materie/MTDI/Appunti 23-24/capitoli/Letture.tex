\chapter{Letture}

\section{\href{https://dighum.ec.tuwien.ac.at/wp-content/uploads/2019/07/Vienna_Manifesto_on_Digital_Humanism_IT.pdf}{Manifesto di vienna per l'umanesimo digitale}}

\paragraph{Il sistema sta fallendo.} Tim Berners-Lee\footnote{Il fondatore del web} sostiene che la digitalizzazione porti con sè vari problemi: perdità di \evidence{privacy}, insorgere di \evidence{comportamenti estremisti}, formazione di \evidence{bolle informative}, etc. Per sostenere l'\glitter{innovazione tecnologica} c'è bisogno di un vasto \glitter{impegno sociale}.

\paragraph{Il manifesto.} Si rivolge alle comunità \evidence{accademiche} e \evidence{professionali}, ai leader \evidence{industriali} e \evidence{politici}.

\paragraph{Tecnologia e società.} I recenti sviluppi tecnologici \newfancyglitter{creano} e \newfancyglitter{distruggono} posti di lavoro e ricchezza. Viene modificata la gerarchia tra \newfancyglitter{uomo} e \newfancyglitter{macchina}.

\paragraph{L'approccio illuminista e umanista.} Il manifesto rappresenta un'estensione della tradizione intellettuale dell'\evidence{umanesimo} che ambisce a creare un'umanità \evidence{illuminata}.

\paragraph{Le tecnologie digitali.} Esse nascono da scelte \glitter{implicite} ed \glitter{esplicite} che promuovono valori, norme, interessi economici, etc.

\paragraph{Umanesimo digitale.} Si deve spostare il \newfancyglitter{focus} dalle tecnologie all'uomo.

\paragraph{I \evidence{princìpi fondamentali} sono:}

\begin{itemize}
    \item Le tecnologie digitali dovrebbero essere progettate per promuovere \\la democrazia
e l‘inclusione;
\item La privacy e la libertà di parola sono valori essenziali per la democrazia e dovrebbero
essere al centro delle nostre attività;
\item Devono essere stabilite norme, regole e leggi efficaci, basate sul dibattito pubblico
e su un ampio consenso;
\item I regolatori devono intervenire sui monopòli tecnologici;
\item Decisioni le cui conseguenze possono influire sui diritti umani individuali o collettivi
devono continuare a essere prese dalle persone;
\item Approcci scientifici interdisciplinari sono un prerequisito per affrontare le sfide
future;
\item Le università sono il luogo in cui si producono nuove conoscenze e si coltiva
il pensiero critico;
\item I ricercatori accademici e industriali devono aprirsi al dialogo con la società e valutare
criticamente i propri approcci;
\item I professionisti di tutto il mondo dovrebbero riconoscere la loro corresponsabilità
nell‘impatto sociale delle tecnologie digitali;
\item È necessaria una visione che consideri nuovi programmi di studio che combinino
la conoscenza delle scienze umane e sociali e degli studi scientifico-ingegneristici;
\item L‘educazione all‘informatica e al suo impatto sociale devono iniziare il prima
possibile.
\end{itemize}

\section{\href{https://link-and-think.blogspot.com/2019/02/informatica-la-terza-rivoluzione-dei-rapporti-di-potere.html}{Informatica: la terza rivoluzione "dei rapporti di potere"}}

L'\evidence{informatica} è una vera e propria rivoluzione per la razza umana, la terza dopo quella della \newfancyglitter{stampa} e quella \newfancyglitter{industriale}.

L'invenzione della stampa nel XV secolo è stata una rivoluzione sia di tipo \evidence{tecnico}\footnote{Si producono testi più velocemente e più economicamente} che di tipo \evidence{sociale}\footnote{Favorisce una maggiore diffusione della conoscenza}. Il potere della \glitter{conoscenza} non è più confinato alle persone che lo posseggono, perchè ogni testo può essere \newfancyglitter{replicato}. La diffusione di testi scientifici, giuridici e letterari hanno portato la società a essere più \evidence{democratica}.

Ed è proprio la conoscenza scientifica\footnote{In primis il metodo Galileiano} a portare alla rivoluzione industriale. A partire dal '700 la disponibilità di \evidence{macchine} rende possibile l'\newfancyglitter{automatizzazione} del lavoro fisico delle persone. Questa rivoluzione ha carattere \evidence{tecnico} perchè permette di ricreare velocemente i manufatti. Inoltre le macchine non si stancano, quindi possono lavorare giorno e notte. Questo mette in discussione il potere della \evidence{natura}: l'uomo la assoggetta e ne superà i limiti.

Nella seconda metà del '900 inizia la terza grande rivoluzione: l'informatica. Essa non è più una replica della conoscenza statica dei libri o della forza fisica delle persone, ma è "\evidence{conoscenza in azione}". Il sapere non è una rappresentazione \newfancyglitter{statica} dei fatti, ma uno scambio di dati \newfancyglitter{interattivo} tra soggetto e realtà. Il potere che viene minato è l'\evidence{intelligenza umana}: essa può essere in qualche modo replicata dai programmi. Basti pensare agli scacchi in cui il computer è in grado di battere un campione del mondo. Oppure ai recenti sviluppi dell'\evidence{intelligenza artificiale}. Tuttavia ci sono due cose che le "macchine cognitive" non sono ancora riuscite a emulare: la \glitter{flessibilità} e l'\glitter{adattabilità}.

In conclusione: è inevitabile il diffondersi di queste nuove tecnologie e dei cambiamenti a loro collegati, ma è importante che ogni individuo sia istruito e formato sulle basi concettuali che permettono di costruire queste macchine.

\pagebreak


\section{\href{https://link-and-think.blogspot.com/2019/03/informatica-e-competenze-digitali-cosa.html}{Informatica e competenze digitali: cosa insegnare?}}

\paragraph{Risposta breve:} entrambe.

\paragraph{Risposta lunga:} questo intero articolo.
\paragraph{}
Nel mondo scolastico c'è confusione tra i termini "\evidence{informatiche}" e "\evidence{digitali}". Purtroppo negli ultimi 30 anni si sono usati in modo intercambiabile questi due termini per riferirsi a: programmare in Pascal, usare Word, scrivere e-mail,  coding, usare i social, etc.

Digitale si riferisce alla rappresentazione di un dato con un simbolo numerico. Informatico si riferisce alla capacità di \evidence{elaborazione automatica} dei dati resa possibile dai metodi e dalle teorie dell'informatica. Usare dei simboli numerici non è una novità, ma lo è elaborare le informazioni in modo automatico, come se si usasse un gigantesco \evidence{orologio}.

L'orologio è solo un esecutore meccanico che non sa che cosa rappresenta o come sia costruito. Il computer quindi può manipolare simboli e istruzione che per lui non hanno significato, ma che lo hanno per l'uomo. Questo è ala base della terza rivoluzione "dei rapporti di potere".

C'è ancora confusione anche sui documenti ufficiali dell'unione europea in cui, per esempio, la programmazione rientrà tra le competenze digitali, sebbene non lo sia. Negli USA o in UK è una competenza informatica. In questi paesi, l'attuale società è semplicemente un'estensione della società \evidence{industriale}. Infatti la società \evidence{digitale} utilizza delle macchine\footnote{Di tipo cognitivo, non meccanico}. Per trasmettere la comprensione della società digitale è necessario portare nelle scuole l'informatica. Ciò non nega l'insegnamento delle competenze digitali che occupano un ruolo chiave nella condivisione della conoscenza.